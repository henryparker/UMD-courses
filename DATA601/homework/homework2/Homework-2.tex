
\documentclass[12pt]{article}
%\documentstyle[12pt]{article}
%\documentclass{amsart}
%\usepackage[dvips]{graphicx}


\usepackage{amssymb,amsmath,amscd,amsthm}
%\usepackage{graphicx,psfrag,epsfig,multirow} LINEA ORIGINAL
\usepackage{graphicx,psfrag,epsfig}


\usepackage{graphicx}
\usepackage[active]{srcltx}

\newtheorem{theorem}{Theorem}[section]
\newtheorem{corollary}[theorem]{Corollary}
\newtheorem{conjecture}[theorem]{Conjecture}
\newtheorem{lemma}[theorem]{Lemma}
%\newtheorem{remark}[theorem]{Remark}
\newtheorem{proposition}[theorem]{Proposition}
\newtheorem{definition}[theorem]{Definition}
\newtheorem{example}[theorem]{Example}
\newtheorem{axiom}{Axiom}
\newtheorem{remark}{Remark}
\newtheorem{exercise}{Exercise}[section]

\newcommand{\thmref}[1]{Theorem~\ref{#1}}
\newcommand{\propref}[1]{Proposition~\ref{#1}}
\newcommand{\secref}[1]{\S\ref{#1}}
\newcommand{\lemref}[1]{Lemma~\ref{#1}}
\newcommand{\corref}[1]{Corollary~\ref{#1}}
\newcommand{\remref}[1]{Remark~\ref{#1}}



\setlength{\topmargin}{0mm}
\setlength{\oddsidemargin}{0mm}
\setlength{\textwidth}{160mm}
\setlength{\textheight}{215mm}
\font\bbc=msbm10 scaled 1200
\newcommand{\E}{\mathbf{E}}
\newcommand{\R}{\mbox {\bbc R}}
\newcommand{\T}{\mbox {\bbc T}}
\newcommand{\Z}{\mbox {\bbc Z}}
\def\stackunder#1#2{\mathrel{\mathop{#2}\limits_{#1}}}

\def\Area{{\rm Area}}
\def\Const{{\rm Const}}
\def\Int{{\rm Int}}

\def\eps{{\varepsilon}}

\def\EXP{\mathbb{E}}
\def\GR{\mathbb{G}}
\def\PROB{\mathbb{P}}
\def\TOR{\mathbb{T}}

\def\naturals{\mathbb{N}}

\def\brGamma{{\bar\Gamma}}
\def\brgamma{{\bar\gamma}}
\def\brtau{{\bar\tau}}
\def\brtheta{{\bar\theta}}
\def\brchi{{\bar\chi}}

\def\bI{{\bf I}}

\def\cE{\mathcal{E}}
\def\cG{\mathcal{C}}
\def\cL{\mathcal{L}}
\def\cU{\mathcal{U}}
\def\cZ{\mathcal{Z}}

\def\hN{{\hat N}}
\def\hn{{\hat n}}
\def\hy{{\hat y}}
\def\hGamma{{\hat\Gamma}}
\def\hdelta{{\hat\delta}}
\def\hsigma{{\hat\sigma}}
\def\htau{{\hat\tau}}
\def\heta{{\hat\eta}}
\def\htheta{{\hat\theta}}

\def\tW{{\tilde W}}
\def\tM{{\tilde M}}
\def\tX{{\tilde X}}
\def\tc{{\tilde c}}
\def\tp{{\tilde p}}
\def\tq{{\tilde q}}
\def\tdelta{{\tilde\delta}}
\def\teta{{\tilde\eta}}
\def\txi{{\tilde\xi}}
\def\tsigma{{\tilde\sigma}}
\def\ttheta{{\tilde\theta}}

\title{Probability and Statistics Homework 2}
\author{Hairui Yin}
\date{}

\begin{document}
%\title{Exam Problems  -  Stat 400}
%\author{Winter 2008-2009}
%\normalsize Department of Mathematics\\[-4pt]
%\normalsize Princeton University\\[-4pt]
%\normalsize Princeton, NJ 08544\\[-4pt]
%\normalsize koralov@math.princeton.edu\\[-4pt]
%\date{}
\maketitle
\noindent {\bf 1.} An elementary school is offering 3 language classes:
one in Spanish, one in French, and one in German. The
classes are open to any of the 100 students in the school.
There are 28 students in the Spanish class, 26 in the French
class, and 16 in the German class. There are 12 students
who are in both Spanish and French, 4 who are in both
Spanish and German, and 6 who are in both French and
German. In addition, there are 2 students taking all 3
classes.

(a) If a student is chosen randomly, what is the probability
that he or she is not in any of the language classes?

(b) If a student is chosen randomly, what is the probability
that he or she is taking exactly one language class?

(c) If 2 students are chosen randomly, what is the probability that at least 1 is taking a language class?
\\
{\bf Answer:}
\\
\textbf{(a)} Denote $S$ as set of students in Spanish class, $F$ as set of students in French class, $G$ as set of students in German class.\\
Given $|S|=28,|F|=26,|G|=16,|S\cap F|=12,|S\cap G|=4, |F\cap G|=6,|S\cap F\cap G|=2$,\\
The number of students in at least one language class is follows:
\begin{align*}
	|S\cup F\cup G| &=|S|+|F|+|G|-|S\cap F|-|S\cap G|-|F\cap G|+|S\cap F\cap G|\\
	&=28+26+16-12-4-6+2\\
	&=50
\end{align*}
So, there are 50 students taking at least on language class.\\
Since there are 100 students in total, the number of students not enrolled in any language class is
\begin{align*}
	100-|S\cup F\cup G|=100-50=50
\end{align*}
Therefore, if a student is chosen randomly, the probability that he/she is not in any of the language class is
$$
P(\text{not in any language class})=\frac{\text{Number of students not in any class}}{\text{Total number of students}}=\frac{50}{100}=\frac{1}{2}
$$
\textbf{(b)} Finding students taking exactly one language class means finding the number of students taking only Spanish, only French, and only German, then summing them together.\\
For Spanish only, denoting as $S_{only}$
\begin{align*}
	S_{only}&=|S\cap F^C\cap G^C|\\
	&=|S\cap(F\cup G)^C|\\
	&=|S| - |S\cap(F\cup G)|\\
	&=|S|-|(S\cap F)\cup (S\cap G)|\\
	&=|S|-(|S\cap F|+|S\cap G|-|S\cap F\cap G|)\\
	&=28-12-4+2\\
	&=14
\end{align*}
Similarly, for French only, denoting as $F_{only}$
\begin{align*}
	F_{only}&=|S^C\cap F\cap G^c|\\
	&=|F|-|S\cap F|-|F\cap G|+|S\cap F\cap G|\\
	&=26-12-6+2\\
	&=10
\end{align*}
Similarly, for German only, denoting as $G_{only}$
\begin{align*}
	G_{only}&=|S^C\cap F^C\cap G|\\
	&=|G|-|S\cap G|-|F\cap G|+|S\cap F\cap G|\\
	&=16-4-6+2\\
	&=8
\end{align*}
So, the total number of students taking exactly one language class is
$$S_{only}+F_{only}+G_{only}=14+10+8=32$$
Therefore, if a student is chosen randomly, the probability that he/she is taking exactly one language class is
$$
P(\text{taking exactly one language class})=\frac{\text{Number of students in exactly one class}}{\text{Total number of students}}=\frac{32}{100}=\frac{8}{25}
$$
\textbf{(c)} Recall that 50 in 100 not in any language class. Considering the case that neither of these two students enroll any language class, the probability is thus
$$P(\text{Neither taking any language class})=\frac{\binom{50}{2}}{\binom{100}{2}}=\frac{49}{198}$$
Therefore, if 2 students are chosen randomly, the probability that at least 1 is taking a language class is
\begin{align*}
	P(\text{at least taking 1 class})&=1-P(\text{Neither taking any language class})\\
	&=1-\frac{49}{198}\\
	&=\frac{149}{198}
\end{align*}
\newpage
\noindent
{\bf 2.} There are eight seats at a rectangular table - three at each of the longer sides and one
at each of the shorter sides. Six adults and two children sit around the table, with each
seating configuration equally likely. What is the probability that the two children will
end up seating next to each other at one of the longer sides of the table?
\\
{\bf Answer:}
\\
Each longer side has three chairs. And two children are sitting next to each other. On one longer sider, there are 2 pairs of adjacent seats. So, on both longer sides, there are 4 pairs of adjacent seats.\\
On each adjacent pair, there are $2!$ arrangements for children. After placing children, there are 6 adults remain. So, the number of ways to arrange these 6 adults are $6!$.\\
Therefore, the probability that the two children will end up seating next to each other at one of the longer sides of the table is
\begin{align*}
	P(\text{Children next to each other at longer table})&=\frac{4\times 2!\times 6!}{8!}\\
	&=\frac{4\times 2}{7\times 8}\\
	&=\frac{1}{7}
\end{align*}
\newpage
\noindent
{\bf 3.}  A man must put 8 balls in a box - some white and some black. He will then pull
out three. How many black balls should he put in the box in order to maximize the
probability that exactly one of the three that he pulls out is black?
\\
{\bf Answer:}
\\
Denote the number of black balls $B$, since there are 8 balls, the number of white balls is $8-B$, $B\in\{0,1,...,7,8\}$.\\
Since pulling out exactly one of three is black, the rest two are white. The probability is as following:
$$P(\text{Exactly 1 black ball})=\frac{\binom{B}{1}\binom{8-B}{2}}{\binom{8}{3}}$$
%Maximizing above probability is equal to maximize $\binom{B}{1}\binom{8-B}{2}$.\\
Notice that if 7 or 8 black balls are put in the box, no way there can be two white balls pulling out. And if 0 black balls is put in, no black balls can be pulled out. So, $B\in\{1,2,3,4,5,6\}$.\\
Considering each case:
\begin{itemize}
	\item[B=1] $P(\text{Exactly 1 black ball})=\frac{7\times 3}{56}=\frac{3}{8}$
	\item[B=2] $P(\text{Exactly 1 black ball})=\frac{2\times 3\times 5}{56}=\frac{15}{28}$
	\item[B=3] $P(\text{Exactly 1 black ball})=\frac{3\times 5\times 2}{56}=\frac{15}{28}$
	\item[B=4] $P(\text{Exactly 1 black ball})=\frac{4\times 2\times 3}{56}=\frac{3}{7}$
	\item[B=5] $P(\text{Exactly 1 black ball})=\frac{5\times 3}{56}=\frac{15}{56}$
	\item[B=6] $P(\text{Exactly 1 black ball})=\frac{6\times 1}{56}=\frac{3}{28}$
\end{itemize}
After comparison, 2 or 3 black balls should be put to maximize the probability that exactly one of three that pulled out is black.
%\\
%\\

\end{document}
