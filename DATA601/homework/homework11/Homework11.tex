
\documentclass[12pt]{article}
%\documentstyle[12pt]{article}
%\documentclass{amsart}
%\usepackage[dvips]{graphicx}


\usepackage{amssymb,amsmath,amscd,amsthm}
%\usepackage{graphicx,psfrag,epsfig,multirow} LINEA ORIGINAL
\usepackage{graphicx,psfrag,epsfig}


\usepackage{graphicx}
\usepackage[active]{srcltx}

\newtheorem{theorem}{Theorem}[section]
\newtheorem{corollary}[theorem]{Corollary}
\newtheorem{conjecture}[theorem]{Conjecture}
\newtheorem{lemma}[theorem]{Lemma}
%\newtheorem{remark}[theorem]{Remark}
\newtheorem{proposition}[theorem]{Proposition}
\newtheorem{definition}[theorem]{Definition}
\newtheorem{example}[theorem]{Example}
\newtheorem{axiom}{Axiom}
\newtheorem{remark}{Remark}
\newtheorem{exercise}{Exercise}[section]

\newcommand{\thmref}[1]{Theorem~\ref{#1}}
\newcommand{\propref}[1]{Proposition~\ref{#1}}
\newcommand{\secref}[1]{\S\ref{#1}}
\newcommand{\lemref}[1]{Lemma~\ref{#1}}
\newcommand{\corref}[1]{Corollary~\ref{#1}}
\newcommand{\remref}[1]{Remark~\ref{#1}}



\setlength{\topmargin}{0mm}
\setlength{\oddsidemargin}{0mm}
\setlength{\textwidth}{160mm}
\setlength{\textheight}{215mm}
\font\bbc=msbm10 scaled 1200
\newcommand{\E}{\mathbf{E}}
\newcommand{\R}{\mbox {\bbc R}}
\newcommand{\T}{\mbox {\bbc T}}
\newcommand{\Z}{\mbox {\bbc Z}}
\def\stackunder#1#2{\mathrel{\mathop{#2}\limits_{#1}}}

\def\Area{{\rm Area}}
\def\Const{{\rm Const}}
\def\Int{{\rm Int}}

\def\eps{{\varepsilon}}

\def\EXP{\mathbb{E}}
\def\GR{\mathbb{G}}
\def\PROB{\mathbb{P}}
\def\TOR{\mathbb{T}}

\def\naturals{\mathbb{N}}

\def\brGamma{{\bar\Gamma}}
\def\brgamma{{\bar\gamma}}
\def\brtau{{\bar\tau}}
\def\brtheta{{\bar\theta}}
\def\brchi{{\bar\chi}}

\def\bI{{\bf I}}

\def\cE{\mathcal{E}}
\def\cG{\mathcal{C}}
\def\cL{\mathcal{L}}
\def\cU{\mathcal{U}}
\def\cZ{\mathcal{Z}}

\def\hN{{\hat N}}
\def\hn{{\hat n}}
\def\hy{{\hat y}}
\def\hGamma{{\hat\Gamma}}
\def\hdelta{{\hat\delta}}
\def\hsigma{{\hat\sigma}}
\def\htau{{\hat\tau}}
\def\heta{{\hat\eta}}
\def\htheta{{\hat\theta}}

\def\tW{{\tilde W}}
\def\tM{{\tilde M}}
\def\tX{{\tilde X}}
\def\tc{{\tilde c}}
\def\tp{{\tilde p}}
\def\tq{{\tilde q}}
\def\tdelta{{\tilde\delta}}
\def\teta{{\tilde\eta}}
\def\txi{{\tilde\xi}}
\def\tsigma{{\tilde\sigma}}
\def\ttheta{{\tilde\theta}}

\title{Probability and Statistics Homework 11}
\author{Hairui Yin}
\date{}

\begin{document}
\maketitle
\noindent{\bf 1.} A trader invented a strategy where he can either win or lose a certain amount of money every day (say, k dollars) with equal probability, independently of all other days. (In fact, such a strategy is not very difficult to implement - buy a very liquid asset whose price moves without major jumps and liquidate the position as soon as you are making or losing k dollars). He then made the following statement: “If I use my strategy every day for the next 400 days, my chances of ending up within 10,000 dollars of the original account balance are equal to 1/3.\\
\indent {(a)} What is k (approximately)?\\
\indent {(b)} What are his chances (approximately) of making over 20,000 with this strategy in 400 days?\\
\\
\textbf{Answer:}\\
\\
\noindent{\textbf{(a)}} For each step, the trader has 50\% winning k dollars and 50\% winning -k dollars, which follows a Bernoulli distribution, with mean 0, variance $k^2$, and $\sigma=k$. Since each round is i.i.d., let $X$ be the total money after the game, $E[X]=0, \sigma_X=\sqrt{n}\sigma=20k$.\\
\begin{align*}
	P(-10000\leq X\leq 10000)&=P(\frac{-10000-0}{20k}\leq \frac{X-\mu}{\sigma_X}\leq \frac{10000-0}{20k})\\
	&=P(\frac{-500}{k}\leq Z\leq \frac{500}{k})
\end{align*}
Consider the z-score of $\Phi(a)-\Phi(-a)=2\Phi(a)-1=\frac{1}{3}$, we can solve that $a\approx 0.43$. Therefore,
\begin{align*}
	\frac{500}{k}\approx0.43\\
	\Rightarrow k\approx1162.79
\end{align*}
\\
\noindent{\textbf{(b)}} The probability of making over 20,000 is 
\begin{align*}
	P(X\geq 20000)&=P(Z\geq \frac{20000-0}{20k})\\
	&=P(Z\geq \frac{20000}{20*1162.79})\\
	&=P(Z\geq 0.86)\\
	&=1-\Phi(0.86)\\
	&=1-0.8051\\
	&=0.1949
\end{align*}
Therefore, the trader has 0.1949 chances to make over 20,000 with his strategy.
\newpage
\noindent{\bf 2.} You are trying to use a machine that only works on some days. If on a given day, the machine is working it will break down the next day with probability $0 < b < 1$, and works on the next day with probability $1 - b$. If it is not working on a given day, it will work
on the next day with probability $0 < r < 1$ and not work the next day with probability $1 - r$.\\
\indent {(a)} In this problem we will formulate this process as a Markov chain. First, let $X^{(t)}$ be a variable that denotes the state of the machine at time t. Then, define a state space S that includes all the possible states that the machine can be in. Lastly, for all $A,B \in S$
find $P(X^{(t+1)} = A \mid X^{(n)} = B)$ ( A and B can be the same state).\\
\indent {(b)} Suppose that on day 1, the machine is working. What is the probability that it is working on day 3?\\
\\
\textbf{Answer:}\\
\\
\noindent{\textbf{(a)}} Let W represent the machine is working, NW represent the machine is not working. The state space is $S=\{W, NW\}$.\\
Given the probability information in the question, for adjacent states, the transition is defined as
\begin{table}[h!]
	\centering
		\begin{tabular}{c|cc}
		   & W & NW \\
		\hline
		W  & $1-b$ & $b$ \\
		NW & $r$ & $1-r$ \\
	\end{tabular}
\end{table}\\
Thus the transition matrix is
$$
Q=
\begin{bmatrix}
	1-b & b \\
	r & 1-r \\ 
\end{bmatrix}
$$
Since $A,B\in S=\{W,NW\}$, we define a mapping function $f$, where $f(S)=1$ if $S=W$, $f(S)=2$ if $S=NW$.\\
Therefore, the probability is
$$P(X^{(t+1)} = A \mid X^{(n)} = B)=(f(B),f(A))^{th}\text{ entry of }Q^{t+1-n}$$
\\
\noindent{\textbf{(b)}} To find this, we can calculate the two-step transition probability $Q^2$.
\begin{align*}
	Q^2&=QQ\\
	&=\begin{bmatrix}
		1-b & b \\
		r & 1-r \\ 
	\end{bmatrix}\times \begin{bmatrix}
		1-b & b \\
		r & 1-r \\ 
	\end{bmatrix}\\
	&=\begin{bmatrix}
		(1-b)^2+br & b(1-b)+b(1-r) \\
		r(1-b)+r^2 & br+(1-r)^2 \\ 
	\end{bmatrix}
\end{align*}
Given the result of question (a), the probability that day 1 machine working and day 3 machine working is $P=(1-b)^2+br$.
\newpage
\noindent{\bf 3.} Suppose that $X_1,...,X_4$ are independent normal random variables with parameters $(a, \sigma^2)$. Find the density of the random variable $Y=X_1-2X_2+3X_3-4X_4$.\\
\\
\textbf{Answer:}\\
\\
$X_1,...,X_4$ are i.i.d. normal random variables. Since we know that $Y$ also follows the normal distribution because its the linear combinition of $X$, we can find it mean and variance.
\begin{align*}
	E[Y]&=E[X_1-2X_2+3X_3-4X_4]\\
	&=E[X_1]-E[2X_2]+E[3X_3]-E[4X_4]\\
	&=E[X_1]-2E[X_2]+3E[X_3]-4E[X_4]\\
	&=a-2a+3a-4a\\
	&=-2a
\end{align*}
\begin{align*}
	Var(Y)&=Var(X_1-2X_2+3X_3-4X_4)\\
	&=Var(X_1)+Var(2X_2)+Var(3X_3)+Var(4X_4)\\
	&=Var(X_1)+2^2Var(X_2)+3^2Var(X_3)+4^2Var(X_4)\\
	&=\sigma^2+4\sigma^2+9\sigma^2+16\sigma^2\\
	&=30\sigma^2
\end{align*}
Therefore, $Y\sim \mathcal{N}(-2a,30\sigma^2)$. The density function is
$$f_Y(y)=\frac{1}{\sqrt{60\pi\sigma^2}}e^{\frac{-(y+2a)^2}{60\sigma^2}}$$
\end{document}