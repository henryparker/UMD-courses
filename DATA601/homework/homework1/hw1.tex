\documentclass[UTF8]{article}
% \usepackage{CTEX}
\usepackage{newtxtext}
\usepackage{geometry}
\usepackage{lipsum} % 该宏包是通过 \lipsum 命令生成一段本文,正式使用时不需要引用该宏包
\usepackage[dvipsnames,svgnames]{xcolor}
\usepackage[strict]{changepage} % 提供一个 adjustwidth 环境
\usepackage{framed} % 实现方框效果
\usepackage{comment} % 注释
\usepackage{xcolor} % 颜色
\usepackage{fancyvrb}
\usepackage{mdframed} % create a gray frame around the symbol
\usepackage{listings} % 定义python代码
\usepackage{tabularx} % 限制长宽的表格
\usepackage{amsmath}
\usepackage{graphicx}
\usepackage{float}
\usepackage{enumitem}
\usepackage{amsthm}
\lstset{
	language=Python,
	basicstyle=\small\ttfamily,
	keywordstyle=\color{blue},
	commentstyle=\color{green},
	stringstyle=\color{red},
	showstringspaces=false,
	numbers=left,
	numberstyle=\tiny,
	breaklines=true,
	frame=single,
	tabsize=2
}


\title{Probability and Statistics Homework 1}
\author{Hairui Yin}
\date{}

\geometry{a4paper,centering,scale=0.8}
% environment derived from framed.sty: see leftbar environment definition
\definecolor{formalshade}{rgb}{0.95,0.95,1} % 文本框颜色
\definecolor{greenshade}{rgb}{0.90,0.99,0.91} % 绿色文本框,竖线颜色设为 Green
\definecolor{redshade}{rgb}{1.00,0.90,0.90}% 红色文本框,竖线颜色设为 LightCoral
\definecolor{brownshade}{rgb}{0.99,0.97,0.93} % 莫兰迪棕色,竖线颜色设为 BurlyWood

% ------------------******-------------------
% 注意行末需要把空格注释掉,不然画出来的方框会有空白竖线

\DefineVerbatimEnvironment{BlueVerb}
	{Verbatim}
	{formatcom=\color{blue}}

%蓝紫色
\newenvironment{Purpoe}{%
	\def\FrameCommand{%
		\hspace{1pt}%
		{\color{DarkBlue}\vrule width 2pt}%
		{\color{formalshade}\vrule width 4pt}%
		\colorbox{formalshade}%
	}%
	\MakeFramed{\advance\hsize-\width\FrameRestore}%
	\noindent\hspace{-4.55pt}% disable indenting first paragraph
	\begin{adjustwidth}{}{7pt}%
		\vspace{2pt}\vspace{2pt}%
	}
	{%
		\vspace{2pt}\end{adjustwidth}\endMakeFramed%
}
% ------------------******-------------------

%绿色
\newenvironment{Green}{%
	\def\FrameCommand{%
		\hspace{1pt}%
		{\color{Green}\vrule width 2pt}%
		{\color{greenshade}\vrule width 4pt}%
		\colorbox{greenshade}%
	}%
	\MakeFramed{\advance\hsize-\width\FrameRestore}%
	\noindent\hspace{-4.55pt}% disable indenting first paragraph
	\begin{adjustwidth}{}{7pt}%
		\vspace{2pt}\vspace{2pt}%
	}
	{%
		\vspace{2pt}\end{adjustwidth}\endMakeFramed%
}
% ------------------******-------------------

%棕色
\newenvironment{Brown}{%
	\def\FrameCommand{%
		\hspace{1pt}%
		{\color{BurlyWood}\vrule width 2pt}%
		{\color{brownshade}\vrule width 4pt}%
		\colorbox{brownshade}%
	}%
	\MakeFramed{\advance\hsize-\width\FrameRestore}%
	\noindent\hspace{-4.55pt}% disable indenting first paragraph
	\begin{adjustwidth}{}{7pt}%
		\vspace{2pt}\vspace{2pt}%
	}
	{%
		\vspace{2pt}\end{adjustwidth}\endMakeFramed%
}
% ------------------******-------------------

%红色
\newenvironment{Red}{%
	\def\FrameCommand{%
		\hspace{1pt}%
		{\color{LightCoral}\vrule width 2pt}%
		{\color{redshade}\vrule width 4pt}%
		\colorbox{redshade}%
	}%
	\MakeFramed{\advance\hsize-\width\FrameRestore}%
	\noindent\hspace{-4.55pt}% disable indenting first paragraph
	\begin{adjustwidth}{}{7pt}%
		\vspace{2pt}\vspace{2pt}%
	}
	{%
		\vspace{2pt}\end{adjustwidth}\endMakeFramed%
}
% ------------------******-------------------

\begin{document}
	\maketitle
	\section*{Problem-1}
	Specifically, the sample space is
	\begin{align*}
		\Omega = \{&(1,1), (1,2), (1,3), (1,4), (1,5), (1,6),\\
				   &(2,1), (2,2), (2,3), (2,4), (2,5), (2,6),\\
				   &(3,1), (3,2), (3,3), (3,4), (3,5), (3,6),\\
				   &(4,1), (4,2), (4,3), (4,4), (4,5), (4,6),\\
				   &(5,1), (5,2), (5,3), (5,4), (5,5), (5,6),\\
				   &(6,1), (6,2), (6,3), (6,4), (6,5), (6,6)\}
	\end{align*}

	Similarly, E, F, G can be written as:
	\begin{align*}
		E=\{&(1,2), (1,4), (1,6), (2,1), (2,3), (2,5), (3,2), (3,4), (3,6),\\
			&(4,1), (4,3), (4,5), (5,2), (5,4), (5,6), (6,1), (6,3), (6,5)\}\\
		F=\{&(1,1), (1,2), (1,3), (1,4), (1,5), (1,6), (2,1), (3,1), (4,1), (5,1), (6,1)\}\\
		G=\{&(1,4), (2,3), (3,2), (4,1))\}
	\end{align*}
	
	With the definition of Union, Intersection and Difference, events are:
	\begin{align*}
		E\cap F=\{&(1,2), (1,4), (1,6), (2,1), (4,1), (6,1)\}\\
		E\cup F=\{&(1,1), (1,2), (1,3), (1,4), (1,5), (1,6), (2,1), (2,3), (2,5),\\
				  &(3,1), (3,2), (3,4), (3,6), (4,1), (4,3), (4,5), (5,1), (5,2),\\
				  &(5,4), (5,6), (6,1), (6,3), (6,5)\}\\
		F\cap G=\{&(1,4), (4,1)\}\\
		E\setminus F =\{&(2,3), (2,5), (3,2), (3,4), (3,6), (4,3), (4,5), (5,2), (5,4), (5,6), (6,3), (6,5)\}\\
		E\cap F\cap G = \{&(1,4), (4,1)\}
	\end{align*}
	
	In conclusion, there are
	\begin{itemize}[itemsep=2pt,topsep=0pt,parsep=0pt]
		\item \textbf{6} elementary outcomes in $E\cap F$
		\item \textbf{23} elementary outcomes in $E\cup F$
		\item \textbf{2} elementary outcomes in $F\cap G$
		\item \textbf{12} elementary outcomes in $E\setminus F$
		\item \textbf{2} elementary outcomes in $E\cap F\cap G$
	\end{itemize}

	\newpage
	\section*{Problem-2}
	
 		When choosing the first sock, the probability that it's red is $\frac{3}{n}$.\\
		After a red sock is picked out, the probability that the second sock is red is $\frac{2}{n-1}$.\\
		Since the probability that both are red is $\frac{1}{2}$, the equation can be listed as
		\begin{equation*}
			\frac{3}{n}\times \frac{2}{n-1}=\frac{1}{2}
		\end{equation*}
		The value of $n$ is $4$.
		
	\section*{Problem-3}
	\begin{proof}
		\begin{enumerate}
			\item According to $\sigma$-algebra property (2), since $A_1,A_2,...\in \mathcal{F}$, then $\Omega\setminus A_1, \Omega\setminus A_2,...\in \mathcal{F}$
			\item According to $\sigma$-algebra property (3), since $\Omega\setminus A_1, \Omega\setminus A_2,...\in \mathcal{F}$, then $\cup^\infty_{i=1}(\Omega\setminus A_i)\in\mathcal{F}$
			\item According to DeMorgan's Law $E^C\cup F^C=(E\cap F)^C$, since
			\begin{equation*}
				\cup^\infty_{i=1}(\Omega\setminus A_i)=\Omega\setminus(\cap^\infty_{i=1}A_i)
			\end{equation*}
			then $\Omega \setminus (\cap^\infty_{i=1}A_i)\in \mathcal{F}$
			\item According to $\sigma$-algebra property (2), since $\Omega \setminus (\cap^\infty_{i=1}A_i)\in \mathcal{F}$, then $\cap^\infty_{i=1}A_i\in \mathcal{F}$
		\end{enumerate}
	\end{proof}
\end{document}