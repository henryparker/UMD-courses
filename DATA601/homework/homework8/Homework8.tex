
\documentclass[12pt]{article}
%\documentstyle[12pt]{article}
%\documentclass{amsart}
%\usepackage[dvips]{graphicx}


\usepackage{amssymb,amsmath,amscd,amsthm}
%\usepackage{graphicx,psfrag,epsfig,multirow} LINEA ORIGINAL
\usepackage{graphicx,psfrag,epsfig}


\usepackage{graphicx}
\usepackage{float}
\usepackage[active]{srcltx}
\usepackage{enumitem}

\newtheorem{theorem}{Theorem}[section]
\newtheorem{corollary}[theorem]{Corollary}
\newtheorem{conjecture}[theorem]{Conjecture}
\newtheorem{lemma}[theorem]{Lemma}
%\newtheorem{remark}[theorem]{Remark}
\newtheorem{proposition}[theorem]{Proposition}
\newtheorem{definition}[theorem]{Definition}
\newtheorem{example}[theorem]{Example}
\newtheorem{axiom}{Axiom}
\newtheorem{remark}{Remark}
\newtheorem{exercise}{Exercise}[section]

\newcommand{\thmref}[1]{Theorem~\ref{#1}}
\newcommand{\propref}[1]{Proposition~\ref{#1}}
\newcommand{\secref}[1]{\S\ref{#1}}
\newcommand{\lemref}[1]{Lemma~\ref{#1}}
\newcommand{\corref}[1]{Corollary~\ref{#1}}
\newcommand{\remref}[1]{Remark~\ref{#1}}



\setlength{\topmargin}{0mm}
\setlength{\oddsidemargin}{0mm}
\setlength{\textwidth}{160mm}
\setlength{\textheight}{215mm}
\font\bbc=msbm10 scaled 1200
\newcommand{\E}{\mathbf{E}}
\newcommand{\R}{\mbox {\bbc R}}
\newcommand{\T}{\mbox {\bbc T}}
\newcommand{\Z}{\mbox {\bbc Z}}
\def\stackunder#1#2{\mathrel{\mathop{#2}\limits_{#1}}}

\def\Area{{\rm Area}}
\def\Const{{\rm Const}}
\def\Int{{\rm Int}}

\def\eps{{\varepsilon}}

\def\EXP{\mathbb{E}}
\def\GR{\mathbb{G}}
\def\PROB{\mathbb{P}}
\def\TOR{\mathbb{T}}

\def\naturals{\mathbb{N}}

\def\brGamma{{\bar\Gamma}}
\def\brgamma{{\bar\gamma}}
\def\brtau{{\bar\tau}}
\def\brtheta{{\bar\theta}}
\def\brchi{{\bar\chi}}

\def\bI{{\bf I}}

\def\cE{\mathcal{E}}
\def\cG{\mathcal{C}}
\def\cL{\mathcal{L}}
\def\cU{\mathcal{U}}
\def\cZ{\mathcal{Z}}

\def\hN{{\hat N}}
\def\hn{{\hat n}}
\def\hy{{\hat y}}
\def\hGamma{{\hat\Gamma}}
\def\hdelta{{\hat\delta}}
\def\hsigma{{\hat\sigma}}
\def\htau{{\hat\tau}}
\def\heta{{\hat\eta}}
\def\htheta{{\hat\theta}}

\def\tW{{\tilde W}}
\def\tM{{\tilde M}}
\def\tX{{\tilde X}}
\def\tc{{\tilde c}}
\def\tp{{\tilde p}}
\def\tq{{\tilde q}}
\def\tdelta{{\tilde\delta}}
\def\teta{{\tilde\eta}}
\def\txi{{\tilde\xi}}
\def\tsigma{{\tilde\sigma}}
\def\ttheta{{\tilde\theta}}

\title{Probability and Statistics Homework 8}
\author{Hairui Yin}
\date{}

\begin{document}
\maketitle
\noindent {\bf 1.} Suppose that $X$ is an exponential random variable with parameter $\lambda=1$. Let $Y=[X]$ (i.e., $Y$ is the integer part of $X$).\\
\indent (a) Write a formula for the probability mass function of $Y$.\\
\indent (b) Calculate $E(Y)$.\\
\\
\textbf{Answer:}\\ 
\noindent {\bf (a)} The pdf of $X$ is
$$f(x)=e^{-x},\quad x\geq 0$$
Since $Y$ is the integer part of $X$, the value of $Y$ is $Y=n$ for $n\leq x\leq n+1, n\in \mathcal{N}$. To find the probability that $Y=n$, we need to compute the probability that $X$ falls in $[n,n+1]$, which is
\begin{align*}
	P(Y=n)&=P(n\leq X\leq n+1)\\
	&=\int_{n}^{n+1}e^{-x}\ dx\\
	&=(1-e^{-1})e^{-n}
\end{align*}
Therefore, the probability mass function of $Y$ is 
$$P(y=n)=(1-e^{-1})e^{-n},\quad n\in \mathcal{N}$$
\noindent {\bf (b)} The expectation of $Y$ is given by
\begin{align*}
	E[Y]&=\sum_{n=0}^{\infty} n(1-e^{-1})e^{-n}\\
	&=(1-e^{-1})\sum_{n=0}^{\infty}ne^{-n}
\end{align*}
Let $S_n=\sum_{i=0}^{n}ie^{-i}=1\cdot e^{-1}+2\cdot e^{-2}+...+n\cdot e^{-n}$. Subtract $S_n$ and $e^{-1}S_n$, we have
$$S_n-e^{-1}S_n=e^{-1}+e^{-2}+...+e^{-n}-n\cdot e^{-(n+1)}$$
Since $e^{-1}+e^{-2}+...+e^{-n}=\frac{e^{-1}(1-e^{-n})}{1-e^{-1}}$, after simplification, $S_n$ is given by
$$S_n=\frac{e^{-1}(1-e^{-n})}{(1-e^{-1})^2}-\frac{ne^{-(n+1)}}{1-e^{-1}}$$
When $n$ goes to infinity, we have
\begin{align*}
	\sum_{n=0}^{\infty}ne^{-n}&=\lim_{n\rightarrow \infty}S_n\\
	&=\lim_{n\rightarrow \infty}\frac{e^{-1}(1-e^{-n})}{(1-e^{-1})^2}-\frac{ne^{-(n+1)}}{1-e^{-1}}\\
	&=\frac{e^{-1}}{(1-e^{-1})^2}
\end{align*}
Therefore, the expectation of $Y$ is given by
\begin{align*}
	E[Y]&=(1-e^{-1})\sum_{n=0}^{\infty}ne^{-n}\\
	&=(1-e^{-1})\frac{e^{-1}}{(1-e^{-1})^2}\\
	&=\frac{e^{-1}}{1-e^{-1}}\\
	&=\frac{1}{e-1}
\end{align*}
\newpage
\noindent {\bf 2.} In the following problem, you should use the normal approximation to the binomial. Use a calculator, a computer program, or an online resource to evaluate (approximately) the required integrals. The answers will be approximate.\\
\indent Supposed that a game is played where you win each round with probability equal to 1/4.\\
\indent (a) If you play 1,000,000 rounds, what is your probability (approximately) to win at least 250,100 rounds?\\
\indent (b) If you play 1,000,000 rounds, find such $n$ that the probability of winning at least $n$ rounds is around 90 percent?\\
\indent (c) How many rounds should you play in order to have your chances of winning 24 percent of the rounds equal to 90 percent?\\
\\
\textbf{Answer:}\\ 
\noindent {\bf (a)} Check if approximation works gives by
$$np(1-p)=1000000\times \frac{1}{4}\times \frac{3}{4}=187500\geq 10$$
Therefore, the approximate probability to win at least 250100 rounds is
\begin{align*}
	P_{Binomial}(S_n\geq 250100)&=P_{Normal}(S_n>250100-0.5)\\
	&=P(\frac{S_n-np}{\sqrt{np(1-p)}}\geq \frac{250100-0.5-np}{\sqrt{np(1-p)}})\\
	&=P(\frac{S_n-1000000\times \frac{1}{4}}{\sqrt{1000000\times\frac{1}{4}(1-\frac{1}{4})}}\geq \frac{250100-0.5-1000000\times \frac{1}{4}}{\sqrt{1000000\times\frac{1}{4}(1-\frac{1}{4})}})\\
	&\approx P(Z\geq 0.22979)\\
	&=1-P(Z\leq 0.22979)\\
	&=1-\Phi(0.22979)\\
	&=0.409
\end{align*}
\\
\noindent {\bf (b)} The probability that at least $n$ rounds is 90 percent is given by
$$P(Z\geq z)=0.9$$
From code, the corresponding z-score $z=-1.2816$. Therefore, the formula is given by
$$\frac{x-0.5-np}{\sqrt{np(1-p)}}=z=-1.2816$$
The above formula gives result $x=249445.5719$. So, the probability of winning at least 249445 rounds is around 90\%.\\
\\
\noindent {\bf (c)} Similar to question (2), the z-score given $P(Z\geq z)=0.9$ is $z=-1.2816$. Let the number of rounds be $N$. Then the formula is given by
$$\frac{0.24N-0.5-Np}{\sqrt{Np(1-p)}}=-1.2816$$
Given the value $p=\frac{1}{4}$, after calculation, the value of $N=2978.61$.\\
Therefore, it should play 2978 rounds to have 90\% chance of winning 24\% of rounds.
\newpage
\noindent {\bf 3.} $X$ and $Y$ are independent random variables uniformly distributed over (0,1). Compute the probability that the larger of the two is at least three times as large as the other one.\\
\\
\textbf{Answer:}\\
The probability that the larger of the two is at least three times larger that the other one is given by
$$P=P(X\geq 3Y)+P(Y\geq 3X)$$
Since $X,Y$ are symmetric, the probability can be simplified to
$$P=2P(X\geq 3Y)$$
Given $X,Y$ are independent uniform R.V., $f(x)=f(y)=1,x\in(0,1),y\in(0,1)$, the probability of $P(X\geq 3Y)$ is thus
\begin{align*}
	P(X\geq 3Y)&=\int_0^1\int_0^{\frac{1}{3}x}f(x)f(y)\ dydx\\
	&=\int_0^1\int_0^{\frac{1}{3}x}1\ dydx\\
	&=\int_0^1\frac{1}{3}x\ dx\\
	&=\frac{1}{6}x^2\big|_0^1\\
	&=\frac{1}{6}
\end{align*}
Therefore, the probability that larger is at least three times than the other is
$$P=2P(X\geq 3Y)=\frac{1}{3}$$
\newpage
\noindent {\bf 4.} The random variables $X$ an $Y$ have joint density
\begin{equation*}
	p(x,y)=
	\left\{
	\begin{aligned}
		&cxy(1-x) &&\text{if}\ 0<x<1,\ 0<y<1,\\
		&0&&\text{otherwise},
	\end{aligned}
	\right.
\end{equation*}
where $c$ is a positive constant.\\
\indent (a) Find $c$.\\
\indent (b) Are $X$ and $Y$ independent?\\
\indent (c) Find E$Y$.\\
\indent (d) Find Var($X$).\\
\\
\textbf{Answer:}\\
\noindent {\bf (a)} The joint density function must integrate to $1$. So,
$$\int_0^1\int_0^1cxy(1-x)\ dydx=1$$
Compute the integral as follows:
\begin{align*}
	\int_0^1\int_0^1cxy(1-x)\ dydx&=\frac{1}{2}c\int_0^1x(1-x)\ dx\\
	&=\frac{1}{2}c(\int_0^1x\ dx-\int_0^1x^2\ dx)\\
	&=\frac{1}{2}c(\frac{1}{2}-\frac{1}{3})\\
	&=\frac{1}{12}c=1
\end{align*}
Therefore, $c=12$.\\
\\
\noindent {\bf (b)} If $p(x,y)=p(x)p(y)$, then $X,Y$ are independent.\\
Firstly, compute $p(x)$.
\begin{align*}
	p(x)&=\int_Yp(x,y)\\
	&=\int_0^1 12xy(1-x)\ dy\\
	&=6x(1-x)
\end{align*}
Then, compute p(y)
\begin{align*}
	p(y)&=\int_Xp(x,y)\\
	&=\int_0^1 12xy(1-x)\ dx\\
	&=2y
\end{align*}
Since $p(x)p(y)=12xy(1-x)=p(x,y)$, $X,Y$ are independent.\\
\\
\noindent {\bf (c)} The expected value of $Y$ is computed from its marginal pdf in question (b)
\begin{align*}
	E[Y]&=\int_0^1yf(y)\ dy\\
	&=\int_0^12y^2\ dy\\
	&=\frac{2}{3}
\end{align*}
\\
\noindent {\bf (d)} The variance value of $X$ can be computed from its marginal pdf in question (b)
\begin{align*}
	Var(X)&=E[X^2]-(E[X])^2\\
	&=\int_0^1x^26x(1-x)\ dx - (\int_0^1x6x(1-x)\ dx)^2\\
	&=6(\int_0^1x^3\ dx-\int_0^1x^4\ dx)-36(\int_0^1x^2\ dx-\int_0^1x^3\ dx)^2\\
	&=6\times \frac{1}{20}-36(\frac{1}{12})^2\\
	&=\frac{3}{10}-\frac{1}{4}\\
	&=\frac{1}{20}
\end{align*}
\end{document}

