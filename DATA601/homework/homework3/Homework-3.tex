
\documentclass[12pt]{article}
%\documentstyle[12pt]{article}
%\documentclass{amsart}
%\usepackage[dvips]{graphicx}


\usepackage{amssymb,amsmath,amscd,amsthm}
%\usepackage{graphicx,psfrag,epsfig,multirow} LINEA ORIGINAL
\usepackage{graphicx,psfrag,epsfig}


\usepackage{graphicx}
\usepackage[active]{srcltx}

\newtheorem{theorem}{Theorem}[section]
\newtheorem{corollary}[theorem]{Corollary}
\newtheorem{conjecture}[theorem]{Conjecture}
\newtheorem{lemma}[theorem]{Lemma}
%\newtheorem{remark}[theorem]{Remark}
\newtheorem{proposition}[theorem]{Proposition}
\newtheorem{definition}[theorem]{Definition}
\newtheorem{example}[theorem]{Example}
\newtheorem{axiom}{Axiom}
\newtheorem{remark}{Remark}
\newtheorem{exercise}{Exercise}[section]

\newcommand{\thmref}[1]{Theorem~\ref{#1}}
\newcommand{\propref}[1]{Proposition~\ref{#1}}
\newcommand{\secref}[1]{\S\ref{#1}}
\newcommand{\lemref}[1]{Lemma~\ref{#1}}
\newcommand{\corref}[1]{Corollary~\ref{#1}}
\newcommand{\remref}[1]{Remark~\ref{#1}}



\setlength{\topmargin}{0mm}
\setlength{\oddsidemargin}{0mm}
\setlength{\textwidth}{160mm}
\setlength{\textheight}{215mm}
\font\bbc=msbm10 scaled 1200
\newcommand{\E}{\mathbf{E}}
\newcommand{\R}{\mbox {\bbc R}}
\newcommand{\T}{\mbox {\bbc T}}
\newcommand{\Z}{\mbox {\bbc Z}}
\def\stackunder#1#2{\mathrel{\mathop{#2}\limits_{#1}}}

\def\Area{{\rm Area}}
\def\Const{{\rm Const}}
\def\Int{{\rm Int}}

\def\eps{{\varepsilon}}

\def\EXP{\mathbb{E}}
\def\GR{\mathbb{G}}
\def\PROB{\mathbb{P}}
\def\TOR{\mathbb{T}}

\def\naturals{\mathbb{N}}

\def\brGamma{{\bar\Gamma}}
\def\brgamma{{\bar\gamma}}
\def\brtau{{\bar\tau}}
\def\brtheta{{\bar\theta}}
\def\brchi{{\bar\chi}}

\def\bI{{\bf I}}

\def\cE{\mathcal{E}}
\def\cG{\mathcal{C}}
\def\cL{\mathcal{L}}
\def\cU{\mathcal{U}}
\def\cZ{\mathcal{Z}}

\def\hN{{\hat N}}
\def\hn{{\hat n}}
\def\hy{{\hat y}}
\def\hGamma{{\hat\Gamma}}
\def\hdelta{{\hat\delta}}
\def\hsigma{{\hat\sigma}}
\def\htau{{\hat\tau}}
\def\heta{{\hat\eta}}
\def\htheta{{\hat\theta}}

\def\tW{{\tilde W}}
\def\tM{{\tilde M}}
\def\tX{{\tilde X}}
\def\tc{{\tilde c}}
\def\tp{{\tilde p}}
\def\tq{{\tilde q}}
\def\tdelta{{\tilde\delta}}
\def\teta{{\tilde\eta}}
\def\txi{{\tilde\xi}}
\def\tsigma{{\tilde\sigma}}
\def\ttheta{{\tilde\theta}}

\title{Probability and Statistics Homework 3}
\author{Hairui Yin}
\date{}

\begin{document}
%\title{Exam Problems  -  Stat 400}
%\author{Winter 2008-2009}
%\normalsize Department of Mathematics\\[-4pt]
%\normalsize Princeton University\\[-4pt]
%\normalsize Princeton, NJ 08544\\[-4pt]
%\normalsize koralov@math.princeton.edu\\[-4pt]
%\date{}
%\maketitle
%{\bf $~~~~~~~~~~~~~~~~~~~~~~~~~~~~~~~~~~$Third Homework}
%\\
%$~$
%\\
\maketitle
\noindent {\bf 1.}  Six people forming three couples sit around a round table (with all the seating configurations equally likely).
What is the probability that, for every couple, the two people forming the couple sit next to each other? (Such a seating arrangement would be convenient if they order three large dishes, each to be shared by one couple.)
\\
{\bf Answer:}
\\
Given the seat number as $d_1,d_2,...,d_6$ in order, where $(d_1,d_2), (d_2,d_3),...,(d_6,d_1)$ are next to each other.\\
Firstly, there are generally 2 ways to divide three groups of couples. Sit in $(d_1,d_2),(d_3,d_4),(d_5,d_6)$, or in $(d_6, d_1),(d_2,d_3),(d_4,d_5)$.
Secondly, to assign three couples into these two ways, each way has $3!$ assignment.\\
Finally, since there are two people in each couple group, each couple group has 2 ways to sit.Since there are three couples, it's $2^3$ ways.\\
Denote that the total arrangement number is $N$, then
$$N=2\times 3!\times 2^3=96$$
The probability that this arrangement happens is
\begin{align*}
	P&=\frac{N}{6!}\\
	&=\frac{96}{720}\\
	&=\frac{2}{15}
\end{align*}


\newpage
\noindent {\bf 2.} Two dice are rolled. 

(a) What is the conditional probability of at least one die showing a 5, given that the sum is larger than 7?

(b) What is the conditional probability of the sum to be larger than 7, given that at least one of the dice is showing a 5?
\\
{\bf Answer:}
\\
\textbf{(a)} Given the sum is larger than 7, the cases are $$(2,6),\textbf{(3,5)},(3,6),(4,4),\textbf{(4,5)},(4,6),\textbf{(5,3)},\textbf{(5,4)},\textbf{(5,5)},\textbf{(5,6)},(6,2),(6,3),(6,4),\textbf{(6,5)},(6,6)$$
Altogether there are 15 cases, in which 7 of them have at least one die showing a 5.
Therefore, the conditional probability of at least one die showing a 5, given that the sum is larger than 7 is
$$P=\frac{7}{15}$$\\
\textbf{(b)} Denote the event that the sum to be larger than 7 is A, the event that at least one of the dice showing a 5 is B. Given the answer of (a), $P(B|A)=\frac{7}{15}$.\\
Consider the probability that the sum is larger than 7, the probability is
$$P(A)=\frac{15}{6\times 6}=\frac{5}{12}$$
Consider the probability that at least one of the dice is showing a 5, it happens when the first die is 5 or the second die is 5, so there are 11 cases. The probability that at least one of the die showing a 5 is thus
$$P(B)=\frac{11}{6\times 6}=\frac{11}{36}$$
According to Bayes's Theorm, the conditional probability of the sum to be larger than 7 given at least one of the dice is a 5 is
\begin{align*}
	P(A|B)&=\frac{P(B|A)P(A)}{P(B)}\\
	&=\frac{\frac{7}{15}\times \frac{5}{12}}{\frac{11}{36}}\\
	&=\frac{7}{11}
\end{align*}
\newpage
\noindent {\bf 3.} Suppose that an ordinary deck of 52 cards is shuffled
and the cards are then turned over one at a time until the
first ace appears. Given that the first ace is the 20th card
to appear, what is the conditional probability that the card
following it is the

(a) ace of spades?

(b) two of clubs?
\\
{\bf Answer:}
\\
\textbf{(a)} Since there are no ace between 1st and 19th card, so after 19th card is droped out, there are 4 aces left. Denote A as event that 20th is ace given 1-19th card is not ace and B as event that 21th is ace of spades, the answer to this quesiton is $P(B|A)$, we can find $P(A\cap B)$ and $P(A)$ to find $P(B|A)$.\\
Considering $P(A\cap B)$, the 20th card is ace but is not spades, and the 21th card should be ace spades, so
$$P(A\cap B)=\frac{3}{(52-19)\times(52-20)}$$
Considering $P(A)$, then
$$P(A)=\frac{4}{52-19}$$
Therefore, the probability that 21th card is ace of spades given the first ace is the 20th card is
\begin{align*}
	P(B|A)&=\frac{P(A\cap B)}{P(A)}\\
	&=\frac{\frac{3}{(52-19)\times(52-20)}}{\frac{4}{52-19}}\\
	&=\frac{3}{128}
\end{align*}
\\
\textbf{(b)} Denote event C as the 21th card is two of clubs, event D as 20th card is the first ace. In order to figure out $P(C|D)$, we need to find $P(C\cap D)$ and $P(D)$.\\
Considering $P(C\cap D)$, the 1-19th card is neither ace or two of clubs, then 20th can be any ace, 21th must be two of clubs. So,
$$P(C\cap D)=\frac{P_{47}^{19}\times 4}{P_{52}^{21}}$$

Considering $P(D)$, the 1-19th cards must not be ace and 20th card is any ace, then
$$P(D)=\frac{P_{48}^{19}\times 4}{P_{52}^{20}}$$
Therefore, the probability that 21th card is two of clubs given the first ace is the 20th card is
\begin{align*}
	P(C|D)&=\frac{P(C\cap D)}{P(D)}\\
	&=\frac{\frac{P_{47}^{19}\times 4}{P_{52}^{21}}}{\frac{P_{48}^{19}\times 4}{P_{52}^{20}}}\\
	&=\frac{29}{1536}
\end{align*}
\newpage
\noindent {\bf 4.} There are three coins in a wallet. The first one is a regular coin - it lands tails
up with probability 50 percent, if tossed. The second is defective - it lands tails up with
probability 60 percent. The third one always lands tails up. A random coin was taken
out of the wallet and tossed. It landed tails up. Given this information, what is the
probability that this was the first coin?
\\
{\bf Answer:}
\\
Denote event A as coin taken out is the first coin, B as the second coin, C as the third coin, D as the coin tails up.\\
The probability that given the coin tails up, it is the first coin is
\begin{align*}
	P(A|D)&=\frac{P(A\cap D)}{P(D)}\\
	&=\frac{P(D|A)P(A)}{P(D|A)P(A)+P(D|B)P(B)+P(D|C)P(C)}\\
	&=\frac{\frac{1}{2}\times \frac{1}{3}}{\frac{1}{2}\times \frac{1}{3}+\frac{3}{5}\times\frac{1}{3}+\frac{1}{3}}\\
	&=\frac{5}{21}
\end{align*}
Therefore, the probability that it is the first coin given tails up is $\frac{5}{21}$.
\end{document}
