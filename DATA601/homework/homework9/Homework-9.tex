
\documentclass[12pt]{article}
%\documentstyle[12pt]{article}
%\documentclass{amsart}
%\usepackage[dvips]{graphicx}


\usepackage{amssymb,amsmath,amscd,amsthm}
%\usepackage{graphicx,psfrag,epsfig,multirow} LINEA ORIGINAL
\usepackage{graphicx,psfrag,epsfig}


\usepackage{graphicx}
\usepackage[active]{srcltx}

\newtheorem{theorem}{Theorem}[section]
\newtheorem{corollary}[theorem]{Corollary}
\newtheorem{conjecture}[theorem]{Conjecture}
\newtheorem{lemma}[theorem]{Lemma}
%\newtheorem{remark}[theorem]{Remark}
\newtheorem{proposition}[theorem]{Proposition}
\newtheorem{definition}[theorem]{Definition}
\newtheorem{example}[theorem]{Example}
\newtheorem{axiom}{Axiom}
\newtheorem{remark}{Remark}
\newtheorem{exercise}{Exercise}[section]

\newcommand{\thmref}[1]{Theorem~\ref{#1}}
\newcommand{\propref}[1]{Proposition~\ref{#1}}
\newcommand{\secref}[1]{\S\ref{#1}}
\newcommand{\lemref}[1]{Lemma~\ref{#1}}
\newcommand{\corref}[1]{Corollary~\ref{#1}}
\newcommand{\remref}[1]{Remark~\ref{#1}}



\setlength{\topmargin}{0mm}
\setlength{\oddsidemargin}{0mm}
\setlength{\textwidth}{160mm}
\setlength{\textheight}{215mm}
\font\bbc=msbm10 scaled 1200
\newcommand{\E}{\mathbf{E}}
\newcommand{\R}{\mbox {\bbc R}}
\newcommand{\T}{\mbox {\bbc T}}
\newcommand{\Z}{\mbox {\bbc Z}}
\def\stackunder#1#2{\mathrel{\mathop{#2}\limits_{#1}}}

\def\Area{{\rm Area}}
\def\Const{{\rm Const}}
\def\Int{{\rm Int}}

\def\eps{{\varepsilon}}

\def\EXP{\mathbb{E}}
\def\GR{\mathbb{G}}
\def\PROB{\mathbb{P}}
\def\TOR{\mathbb{T}}

\def\naturals{\mathbb{N}}

\def\brGamma{{\bar\Gamma}}
\def\brgamma{{\bar\gamma}}
\def\brtau{{\bar\tau}}
\def\brtheta{{\bar\theta}}
\def\brchi{{\bar\chi}}

\def\bI{{\bf I}}

\def\cE{\mathcal{E}}
\def\cG{\mathcal{C}}
\def\cL{\mathcal{L}}
\def\cU{\mathcal{U}}
\def\cZ{\mathcal{Z}}

\def\hN{{\hat N}}
\def\hn{{\hat n}}
\def\hy{{\hat y}}
\def\hGamma{{\hat\Gamma}}
\def\hdelta{{\hat\delta}}
\def\hsigma{{\hat\sigma}}
\def\htau{{\hat\tau}}
\def\heta{{\hat\eta}}
\def\htheta{{\hat\theta}}

\def\tW{{\tilde W}}
\def\tM{{\tilde M}}
\def\tX{{\tilde X}}
\def\tc{{\tilde c}}
\def\tp{{\tilde p}}
\def\tq{{\tilde q}}
\def\tdelta{{\tilde\delta}}
\def\teta{{\tilde\eta}}
\def\txi{{\tilde\xi}}
\def\tsigma{{\tilde\sigma}}
\def\ttheta{{\tilde\theta}}

\title{Probability and Statistics Homework 9}
\author{Hairui Yin}
\date{}

\begin{document}
\maketitle
\noindent{\bf 1.} Suppose that there are two identically-looking batteries in a box. The first one should last for a time that is
exponentially distributed with parameter $\lambda_1 = 1$ (in months). The second one should last for a time that is
exponentially distributed with parameter $\lambda_2 = 2$. A battery has been picked randomly and is still working after two months. 
Given this information, what is the probability that the first battery was picked?
\\
\\
\textbf{Answer:}\\
Denote the event that the first battery is picked to be $A_1$, the second one is picked to be $A_2$, the battery works longer than 2 months to be $L$. The problem is to find the probability that the first battery was picked given its working after two months, which is $$P(A_1|L)$$
According to Bayes Rule, the formula can be rewritten as
\begin{align*}
	P(A_1|L)&=\frac{P(A_1\cap L)}{P(L)}\\
	&=\frac{P(L|A_1)P(A_1)}{P(L|A_1)P(A_1)+P(L|A_2)P(A_2)}\\
	&=\frac{P(T>2|\lambda_1=1)\frac{1}{2}}{P(T>2|\lambda_1=1)\frac{1}{2}+P(T>2|\lambda_2=2)\frac{1}{2}}\\
\end{align*}
Given the cdf of exponential distribution is $F(x)=1-e^{-\lambda x}$, the above formula is given be
\begin{align*}
	P(A_1|L)&=\frac{e^{-2}}{e^{-2}+e^{-4}}\\
	&=\frac{e^2}{1+e^2}
\end{align*}
Therefore, the probability that the first battery was picked given it's still working after two months is $\frac{e^2}{1+e^2}$.
\newpage
\noindent{\bf 2.}  Suppose that ${\rm Var}(X) = 3$, ${\rm Var}(Y) = 4$, and ${\rm Cov}(X, Y) = 1$. 

(a) Find ${\rm Cov}(2X -Y, X + 3Y)$.

(b) Find $\rho(2X -Y, X + 3Y)$ (the correlation between $2X -Y$ and $X + 3Y$).
\\
\\
\textbf{Answer:}\\
\\
\noindent {\bf (a)} With the property of Covirance calculation, we have
\begin{align*}
	Cov(2X-Y,X+3Y)&=Cov(2X-Y,X)+Cov(2X-Y,3Y)\\
	&=2Cov(X,X)-Cov(X,Y)+6Cov(X,Y)-3Cov(Y)\\
	&=2Var(X)+5Cov(X,Y)-3Var(Y)\\
	&=2\times 3 + 5\times 1-3\times 4\\
	&=-1
\end{align*}
\\
\noindent {\bf (b)} According to the definition of correlation $\rho$,
\begin{align*}
	\rho(2X-Y,X+3Y)&=\frac{Cov(2X-Y,X+3Y)}{\sqrt{Var(2X-Y)}\sqrt{Var(X+3Y)}}\\
	&=\frac{Cov(2X-Y,X+3Y)}{\sqrt{4Var(X)-4Cov(X,Y)+Var(Y)}\sqrt{Var(X)+6Cov(X,Y)+9Var(Y)}}\\
	&=\frac{-1}{\sqrt{4\times 3-4\times 1+4}\sqrt{3+6\times 1+9\times 4}}\\
	&=-\frac{\sqrt{15}}{90}
\end{align*}
\newpage
\noindent{\bf 3.} Suppose that $X$ and $Y$ are independent random variables. Both are exponential with parameter 1. Find the density of the random
variable $X+Y$.
\\
\\
\textbf{Answer:}\\
Given two independent random variable $X,Y$, the pdf of $X+Y$ is the convolution
$$f_{X+Y}(a)=\int_{-\infty}^{\infty}f_X(a-y)f_Y(y)\ dy$$
the density of the random variable $Z=X+Y$ is
\begin{align*}
	f_{Z}(z)&=\int_0^zf_X(z-y)f_Y(y)\ dy\\
	&=\int_0^ze^{-(z-y)}e^{-y}\ dy\\
	&=e^{-z}\int_0^z\ dy\\
	&=ze^{-z}
\end{align*}
Therefore, the pdf of $Z$ is given by
\begin{equation*}
	f_Z(z)=\left\{
	\begin{aligned}
		&ze^{-z},&&z\geq 0\\
		&0,&&z<0
	\end{aligned}
	\right.
\end{equation*}
\\

\end{document}